% Options for packages loaded elsewhere
\PassOptionsToPackage{unicode}{hyperref}
\PassOptionsToPackage{hyphens}{url}
\PassOptionsToPackage{dvipsnames,svgnames,x11names}{xcolor}
%
\documentclass[
  letterpaper,
  DIV=11,
  numbers=noendperiod]{scrartcl}

\usepackage{amsmath,amssymb}
\usepackage{iftex}
\ifPDFTeX
  \usepackage[T1]{fontenc}
  \usepackage[utf8]{inputenc}
  \usepackage{textcomp} % provide euro and other symbols
\else % if luatex or xetex
  \usepackage{unicode-math}
  \defaultfontfeatures{Scale=MatchLowercase}
  \defaultfontfeatures[\rmfamily]{Ligatures=TeX,Scale=1}
\fi
\usepackage{lmodern}
\ifPDFTeX\else  
    % xetex/luatex font selection
\fi
% Use upquote if available, for straight quotes in verbatim environments
\IfFileExists{upquote.sty}{\usepackage{upquote}}{}
\IfFileExists{microtype.sty}{% use microtype if available
  \usepackage[]{microtype}
  \UseMicrotypeSet[protrusion]{basicmath} % disable protrusion for tt fonts
}{}
\makeatletter
\@ifundefined{KOMAClassName}{% if non-KOMA class
  \IfFileExists{parskip.sty}{%
    \usepackage{parskip}
  }{% else
    \setlength{\parindent}{0pt}
    \setlength{\parskip}{6pt plus 2pt minus 1pt}}
}{% if KOMA class
  \KOMAoptions{parskip=half}}
\makeatother
\usepackage{xcolor}
\setlength{\emergencystretch}{3em} % prevent overfull lines
\setcounter{secnumdepth}{-\maxdimen} % remove section numbering
% Make \paragraph and \subparagraph free-standing
\ifx\paragraph\undefined\else
  \let\oldparagraph\paragraph
  \renewcommand{\paragraph}[1]{\oldparagraph{#1}\mbox{}}
\fi
\ifx\subparagraph\undefined\else
  \let\oldsubparagraph\subparagraph
  \renewcommand{\subparagraph}[1]{\oldsubparagraph{#1}\mbox{}}
\fi


\providecommand{\tightlist}{%
  \setlength{\itemsep}{0pt}\setlength{\parskip}{0pt}}\usepackage{longtable,booktabs,array}
\usepackage{calc} % for calculating minipage widths
% Correct order of tables after \paragraph or \subparagraph
\usepackage{etoolbox}
\makeatletter
\patchcmd\longtable{\par}{\if@noskipsec\mbox{}\fi\par}{}{}
\makeatother
% Allow footnotes in longtable head/foot
\IfFileExists{footnotehyper.sty}{\usepackage{footnotehyper}}{\usepackage{footnote}}
\makesavenoteenv{longtable}
\usepackage{graphicx}
\makeatletter
\def\maxwidth{\ifdim\Gin@nat@width>\linewidth\linewidth\else\Gin@nat@width\fi}
\def\maxheight{\ifdim\Gin@nat@height>\textheight\textheight\else\Gin@nat@height\fi}
\makeatother
% Scale images if necessary, so that they will not overflow the page
% margins by default, and it is still possible to overwrite the defaults
% using explicit options in \includegraphics[width, height, ...]{}
\setkeys{Gin}{width=\maxwidth,height=\maxheight,keepaspectratio}
% Set default figure placement to htbp
\makeatletter
\def\fps@figure{htbp}
\makeatother

\KOMAoption{captions}{tableheading}
\makeatletter
\@ifpackageloaded{tcolorbox}{}{\usepackage[skins,breakable]{tcolorbox}}
\@ifpackageloaded{fontawesome5}{}{\usepackage{fontawesome5}}
\definecolor{quarto-callout-color}{HTML}{909090}
\definecolor{quarto-callout-note-color}{HTML}{0758E5}
\definecolor{quarto-callout-important-color}{HTML}{CC1914}
\definecolor{quarto-callout-warning-color}{HTML}{EB9113}
\definecolor{quarto-callout-tip-color}{HTML}{00A047}
\definecolor{quarto-callout-caution-color}{HTML}{FC5300}
\definecolor{quarto-callout-color-frame}{HTML}{acacac}
\definecolor{quarto-callout-note-color-frame}{HTML}{4582ec}
\definecolor{quarto-callout-important-color-frame}{HTML}{d9534f}
\definecolor{quarto-callout-warning-color-frame}{HTML}{f0ad4e}
\definecolor{quarto-callout-tip-color-frame}{HTML}{02b875}
\definecolor{quarto-callout-caution-color-frame}{HTML}{fd7e14}
\makeatother
\makeatletter
\makeatother
\makeatletter
\makeatother
\makeatletter
\@ifpackageloaded{caption}{}{\usepackage{caption}}
\AtBeginDocument{%
\ifdefined\contentsname
  \renewcommand*\contentsname{Table des matières}
\else
  \newcommand\contentsname{Table des matières}
\fi
\ifdefined\listfigurename
  \renewcommand*\listfigurename{Liste des Figures}
\else
  \newcommand\listfigurename{Liste des Figures}
\fi
\ifdefined\listtablename
  \renewcommand*\listtablename{Liste des Tables}
\else
  \newcommand\listtablename{Liste des Tables}
\fi
\ifdefined\figurename
  \renewcommand*\figurename{Figure}
\else
  \newcommand\figurename{Figure}
\fi
\ifdefined\tablename
  \renewcommand*\tablename{Table}
\else
  \newcommand\tablename{Table}
\fi
}
\@ifpackageloaded{float}{}{\usepackage{float}}
\floatstyle{ruled}
\@ifundefined{c@chapter}{\newfloat{codelisting}{h}{lop}}{\newfloat{codelisting}{h}{lop}[chapter]}
\floatname{codelisting}{Listing}
\newcommand*\listoflistings{\listof{codelisting}{Liste des Listings}}
\makeatother
\makeatletter
\@ifpackageloaded{caption}{}{\usepackage{caption}}
\@ifpackageloaded{subcaption}{}{\usepackage{subcaption}}
\makeatother
\makeatletter
\makeatother
\ifLuaTeX
\usepackage[bidi=basic]{babel}
\else
\usepackage[bidi=default]{babel}
\fi
\babelprovide[main,import]{french}
% get rid of language-specific shorthands (see #6817):
\let\LanguageShortHands\languageshorthands
\def\languageshorthands#1{}
\ifLuaTeX
  \usepackage{selnolig}  % disable illegal ligatures
\fi
\IfFileExists{bookmark.sty}{\usepackage{bookmark}}{\usepackage{hyperref}}
\IfFileExists{xurl.sty}{\usepackage{xurl}}{} % add URL line breaks if available
\urlstyle{same} % disable monospaced font for URLs
\hypersetup{
  pdftitle={R avec VScode},
  pdfauthor={Coralie Cottet},
  pdflang={fr},
  colorlinks=true,
  linkcolor={blue},
  filecolor={Maroon},
  citecolor={Blue},
  urlcolor={Blue},
  pdfcreator={LaTeX via pandoc}}

\title{R avec VScode}
\author{Coralie Cottet}
\date{2023-06-07}

\begin{document}
\maketitle
\begin{abstract}
VSCode est un IDE (\textbf{Environnement de Dévelopement Integré}) qui
offre une intégration avec de nombreux outils et langages de
programmation. L'extension R pour VS Code vous permet de travailler avec
R dans un environnement de développement intégré, ce qui peut vous faire
gagner du temps. VS Code vous permet de personnaliser l'éditeur de code
et les paramètres de l'IDE selon vos besoins.
\end{abstract}
\begin{figure}

{\centering 

\href{https://code.visualstudio.com/}{\includegraphics[width=0.1\textwidth,height=\textheight]{R_vscode_doc_files/mediabag/Visual_Studio_Code_0.png}}

}

\end{figure}

\hypertarget{pas-uxe0-pas}{%
\section{Pas à pas}\label{pas-uxe0-pas}}

\textbf{Installation de VScode}

L'installation de vscode se fait sur le site suivant. Selectionner le
bon système d'exploitation et télécharger VSCode.

\url{https://code.visualstudio.com/Download}

\begin{tcolorbox}[enhanced jigsaw, colframe=quarto-callout-tip-color-frame, toprule=.15mm, titlerule=0mm, opacityback=0, bottomtitle=1mm, bottomrule=.15mm, rightrule=.15mm, title=\textcolor{quarto-callout-tip-color}{\faLightbulb}\hspace{0.5em}{A l'Ined}, leftrule=.75mm, colbacktitle=quarto-callout-tip-color!10!white, toptitle=1mm, arc=.35mm, colback=white, opacitybacktitle=0.6, breakable, left=2mm, coltitle=black]

VScode est intégré à \textbf{Applined}, et peut donc être directement
installé via cette application.

\end{tcolorbox}

\textbf{Installation de l'extension R}

Maintenant que vous êtes sur VSCode, il faut cliquer sur l'icône situé
sur le côté gauche :

\includegraphics{img/icone_extension_vscode.png}

\begin{tcolorbox}[enhanced jigsaw, colframe=quarto-callout-note-color-frame, toprule=.15mm, titlerule=0mm, opacityback=0, bottomtitle=1mm, bottomrule=.15mm, rightrule=.15mm, title=\textcolor{quarto-callout-note-color}{\faInfo}\hspace{0.5em}{Pour Coralie (insertion d'image)}, leftrule=.75mm, colbacktitle=quarto-callout-note-color!10!white, toptitle=1mm, arc=.35mm, colback=white, opacitybacktitle=0.6, breakable, left=2mm, coltitle=black]

\begin{itemize}
\tightlist
\item
  Plus facile d'utiliser \texttt{!{[}{]}(icone\_extension\_vscode.png)}
\item
  Tu peux ajouter des options type réduire la taille
  \texttt{!{[}{]}(icone\_extension\_vscode.png)\{width=70\%\}}
\item
  Ca permet d'appliquer des extensions à l'image, ici lightbox (voir
  captures plus bas)
\item
  La balise html classique ne permet pas l'exportation en format pdf
\end{itemize}

\end{tcolorbox}

Dans la barre de recherche il faut inscrire R, puis cliquer sur
télécharger. Télécharger également l'extension \texttt{Rtools} de la
même manière que la précédente. Ensuite ouvrez un fichier en cliquant
sur file, puis \emph{new file} et selectionnez le langage dans lequel
vous souhaitez coder (ici R).

\textbf{Comment executer le programme?}

Il suffit de cliquer sur le bouton ``run'':

\includegraphics{img/run_r_vscode.png}

\textbf{Comment exécuter une cellule ?}

Il suffit de se mettre sur la ligne de code à exécuter et de faire
\textbf{\texttt{crtl+entrée}}.

\begin{tcolorbox}[enhanced jigsaw, colframe=quarto-callout-important-color-frame, toprule=.15mm, titlerule=0mm, opacityback=0, bottomtitle=1mm, bottomrule=.15mm, rightrule=.15mm, title=\textcolor{quarto-callout-important-color}{\faExclamation}\hspace{0.5em}{Possible conflit entre les extensions R et Rtools}, leftrule=.75mm, colbacktitle=quarto-callout-important-color!10!white, toptitle=1mm, arc=.35mm, colback=white, opacitybacktitle=0.6, breakable, left=2mm, coltitle=black]

Si cela ne marche pas il peut avoir une erreur de connection entre
l'extension Rtools et R. (erreur commune : \emph{{[}Error - 2:58:07
PM{]} R Tools client: couldn't create connection to server.}) Pour la
résoudre: \textbf{\texttt{Crtl+Shift+P}}. Cliquez sur préférence :
\textbf{\emph{Open Keyboard shortcut et supprimez r.execute in
terminal}}.

Lien de l'explication de résolution d'erreur:

\href{https://stackoverflow.com/questions/75261815/r-tools-client-couldnt-create-connection-to-server-launching-serverng-com}{Lien
vers la résolution de l'erreur}

\end{tcolorbox}

\hypertarget{captures-duxe9cran}{%
\section{Captures d'écran}\label{captures-duxe9cran}}

\begin{tcolorbox}[enhanced jigsaw, colframe=quarto-callout-note-color-frame, toprule=.15mm, titlerule=0mm, opacityback=0, bottomtitle=1mm, bottomrule=.15mm, rightrule=.15mm, title=\textcolor{quarto-callout-note-color}{\faInfo}\hspace{0.5em}{Note}, leftrule=.75mm, colbacktitle=quarto-callout-note-color!10!white, toptitle=1mm, arc=.35mm, colback=white, opacitybacktitle=0.6, breakable, left=2mm, coltitle=black]

Les captures ne sont disponibles que pour le format html

\end{tcolorbox}



\end{document}
