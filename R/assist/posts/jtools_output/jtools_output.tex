% Options for packages loaded elsewhere
\PassOptionsToPackage{unicode}{hyperref}
\PassOptionsToPackage{hyphens}{url}
\PassOptionsToPackage{dvipsnames,svgnames,x11names}{xcolor}
%
\documentclass[
  letterpaper,
  DIV=11,
  numbers=noendperiod]{scrartcl}

\usepackage{amsmath,amssymb}
\usepackage{iftex}
\ifPDFTeX
  \usepackage[T1]{fontenc}
  \usepackage[utf8]{inputenc}
  \usepackage{textcomp} % provide euro and other symbols
\else % if luatex or xetex
  \usepackage{unicode-math}
  \defaultfontfeatures{Scale=MatchLowercase}
  \defaultfontfeatures[\rmfamily]{Ligatures=TeX,Scale=1}
\fi
\usepackage{lmodern}
\ifPDFTeX\else  
    % xetex/luatex font selection
\fi
% Use upquote if available, for straight quotes in verbatim environments
\IfFileExists{upquote.sty}{\usepackage{upquote}}{}
\IfFileExists{microtype.sty}{% use microtype if available
  \usepackage[]{microtype}
  \UseMicrotypeSet[protrusion]{basicmath} % disable protrusion for tt fonts
}{}
\makeatletter
\@ifundefined{KOMAClassName}{% if non-KOMA class
  \IfFileExists{parskip.sty}{%
    \usepackage{parskip}
  }{% else
    \setlength{\parindent}{0pt}
    \setlength{\parskip}{6pt plus 2pt minus 1pt}}
}{% if KOMA class
  \KOMAoptions{parskip=half}}
\makeatother
\usepackage{xcolor}
\setlength{\emergencystretch}{3em} % prevent overfull lines
\setcounter{secnumdepth}{-\maxdimen} % remove section numbering
% Make \paragraph and \subparagraph free-standing
\ifx\paragraph\undefined\else
  \let\oldparagraph\paragraph
  \renewcommand{\paragraph}[1]{\oldparagraph{#1}\mbox{}}
\fi
\ifx\subparagraph\undefined\else
  \let\oldsubparagraph\subparagraph
  \renewcommand{\subparagraph}[1]{\oldsubparagraph{#1}\mbox{}}
\fi

\usepackage{color}
\usepackage{fancyvrb}
\newcommand{\VerbBar}{|}
\newcommand{\VERB}{\Verb[commandchars=\\\{\}]}
\DefineVerbatimEnvironment{Highlighting}{Verbatim}{commandchars=\\\{\}}
% Add ',fontsize=\small' for more characters per line
\usepackage{framed}
\definecolor{shadecolor}{RGB}{40,42,54}
\newenvironment{Shaded}{\begin{snugshade}}{\end{snugshade}}
\newcommand{\AlertTok}[1]{\textcolor[rgb]{1.00,0.33,0.33}{\textbf{#1}}}
\newcommand{\AnnotationTok}[1]{\textcolor[rgb]{1.00,0.47,0.78}{#1}}
\newcommand{\AttributeTok}[1]{\textcolor[rgb]{1.00,0.47,0.78}{#1}}
\newcommand{\BaseNTok}[1]{\textcolor[rgb]{0.74,0.58,0.98}{#1}}
\newcommand{\BuiltInTok}[1]{\textcolor[rgb]{0.55,0.91,0.99}{#1}}
\newcommand{\CharTok}[1]{\textcolor[rgb]{0.95,0.98,0.55}{#1}}
\newcommand{\CommentTok}[1]{\textcolor[rgb]{0.38,0.45,0.64}{#1}}
\newcommand{\CommentVarTok}[1]{\textcolor[rgb]{0.55,0.91,0.99}{#1}}
\newcommand{\ConstantTok}[1]{\textcolor[rgb]{0.74,0.58,0.98}{\textbf{#1}}}
\newcommand{\ControlFlowTok}[1]{\textcolor[rgb]{1.00,0.47,0.78}{#1}}
\newcommand{\DataTypeTok}[1]{\textcolor[rgb]{0.55,0.91,0.99}{\textit{#1}}}
\newcommand{\DecValTok}[1]{\textcolor[rgb]{0.74,0.58,0.98}{#1}}
\newcommand{\DocumentationTok}[1]{\textcolor[rgb]{1.00,0.72,0.42}{#1}}
\newcommand{\ErrorTok}[1]{\textcolor[rgb]{1.00,0.33,0.33}{\underline{#1}}}
\newcommand{\ExtensionTok}[1]{\textcolor[rgb]{0.55,0.91,0.99}{#1}}
\newcommand{\FloatTok}[1]{\textcolor[rgb]{0.74,0.58,0.98}{#1}}
\newcommand{\FunctionTok}[1]{\textcolor[rgb]{0.31,0.98,0.48}{#1}}
\newcommand{\ImportTok}[1]{\textcolor[rgb]{1.00,0.47,0.78}{#1}}
\newcommand{\InformationTok}[1]{\textcolor[rgb]{0.95,0.98,0.55}{#1}}
\newcommand{\KeywordTok}[1]{\textcolor[rgb]{1.00,0.47,0.78}{#1}}
\newcommand{\NormalTok}[1]{\textcolor[rgb]{0.97,0.97,0.95}{#1}}
\newcommand{\OperatorTok}[1]{\textcolor[rgb]{0.97,0.97,0.95}{#1}}
\newcommand{\OtherTok}[1]{\textcolor[rgb]{0.31,0.98,0.48}{#1}}
\newcommand{\PreprocessorTok}[1]{\textcolor[rgb]{1.00,0.47,0.78}{#1}}
\newcommand{\RegionMarkerTok}[1]{\textcolor[rgb]{0.55,0.91,0.99}{#1}}
\newcommand{\SpecialCharTok}[1]{\textcolor[rgb]{1.00,0.47,0.78}{#1}}
\newcommand{\SpecialStringTok}[1]{\textcolor[rgb]{0.95,0.98,0.55}{#1}}
\newcommand{\StringTok}[1]{\textcolor[rgb]{0.95,0.98,0.55}{#1}}
\newcommand{\VariableTok}[1]{\textcolor[rgb]{0.55,0.91,0.99}{#1}}
\newcommand{\VerbatimStringTok}[1]{\textcolor[rgb]{0.95,0.98,0.55}{#1}}
\newcommand{\WarningTok}[1]{\textcolor[rgb]{1.00,0.33,0.33}{#1}}

\providecommand{\tightlist}{%
  \setlength{\itemsep}{0pt}\setlength{\parskip}{0pt}}\usepackage{longtable,booktabs,array}
\usepackage{calc} % for calculating minipage widths
% Correct order of tables after \paragraph or \subparagraph
\usepackage{etoolbox}
\makeatletter
\patchcmd\longtable{\par}{\if@noskipsec\mbox{}\fi\par}{}{}
\makeatother
% Allow footnotes in longtable head/foot
\IfFileExists{footnotehyper.sty}{\usepackage{footnotehyper}}{\usepackage{footnote}}
\makesavenoteenv{longtable}
\usepackage{graphicx}
\makeatletter
\def\maxwidth{\ifdim\Gin@nat@width>\linewidth\linewidth\else\Gin@nat@width\fi}
\def\maxheight{\ifdim\Gin@nat@height>\textheight\textheight\else\Gin@nat@height\fi}
\makeatother
% Scale images if necessary, so that they will not overflow the page
% margins by default, and it is still possible to overwrite the defaults
% using explicit options in \includegraphics[width, height, ...]{}
\setkeys{Gin}{width=\maxwidth,height=\maxheight,keepaspectratio}
% Set default figure placement to htbp
\makeatletter
\def\fps@figure{htbp}
\makeatother

\KOMAoption{captions}{tableheading}
\makeatletter
\makeatother
\makeatletter
\makeatother
\makeatletter
\@ifpackageloaded{caption}{}{\usepackage{caption}}
\AtBeginDocument{%
\ifdefined\contentsname
  \renewcommand*\contentsname{Table des matières}
\else
  \newcommand\contentsname{Table des matières}
\fi
\ifdefined\listfigurename
  \renewcommand*\listfigurename{Liste des Figures}
\else
  \newcommand\listfigurename{Liste des Figures}
\fi
\ifdefined\listtablename
  \renewcommand*\listtablename{Liste des Tables}
\else
  \newcommand\listtablename{Liste des Tables}
\fi
\ifdefined\figurename
  \renewcommand*\figurename{Figure}
\else
  \newcommand\figurename{Figure}
\fi
\ifdefined\tablename
  \renewcommand*\tablename{Table}
\else
  \newcommand\tablename{Table}
\fi
}
\@ifpackageloaded{float}{}{\usepackage{float}}
\floatstyle{ruled}
\@ifundefined{c@chapter}{\newfloat{codelisting}{h}{lop}}{\newfloat{codelisting}{h}{lop}[chapter]}
\floatname{codelisting}{Listing}
\newcommand*\listoflistings{\listof{codelisting}{Liste des Listings}}
\makeatother
\makeatletter
\@ifpackageloaded{caption}{}{\usepackage{caption}}
\@ifpackageloaded{subcaption}{}{\usepackage{subcaption}}
\makeatother
\makeatletter
\makeatother
\makeatletter
\@ifpackageloaded{tikz}{}{\usepackage{tikz}}
\makeatother
        \newcommand*\circled[1]{\tikz[baseline=(char.base)]{
          \node[shape=circle,draw,inner sep=1pt] (char) {{\scriptsize#1}};}}  
                  
\ifLuaTeX
\usepackage[bidi=basic]{babel}
\else
\usepackage[bidi=default]{babel}
\fi
\babelprovide[main,import]{french}
% get rid of language-specific shorthands (see #6817):
\let\LanguageShortHands\languageshorthands
\def\languageshorthands#1{}
\ifLuaTeX
  \usepackage{selnolig}  % disable illegal ligatures
\fi
\IfFileExists{bookmark.sty}{\usepackage{bookmark}}{\usepackage{hyperref}}
\IfFileExists{xurl.sty}{\usepackage{xurl}}{} % add URL line breaks if available
\urlstyle{same} % disable monospaced font for URLs
\hypersetup{
  pdftitle={Affichage d'output de régression avec Jtools},
  pdfauthor={Marc Thévenin},
  pdflang={fr},
  colorlinks=true,
  linkcolor={blue},
  filecolor={Maroon},
  citecolor={Blue},
  urlcolor={Blue},
  pdfcreator={LaTeX via pandoc}}

\title{Affichage d'output de régression avec Jtools}
\author{Marc Thévenin}
\date{2023-06-21}

\begin{document}
\maketitle
\begin{abstract}
La fonction \texttt{summ} du package jtools (Jacob Long) permet
d'obtenir des outputs de regression en format console de très bonne
qualité. Il s'agit d'une très bonne alternative aux outputs par défaut.
Ce package propose également des fonctionnalités d'exportation des
outputs dans d'autres formats (html, docx, pdf \ldots) et des
visualisations sous forme de graphique. Ces fonctionnalité ne seront pas
traitées ici, l'accent étant mis sur la qualité d'un output console avec
une durée d'exécution minimale. L'utilisation des fonctionnalités
d'exportation, comme le très populaire package
\textbf{\texttt{gtsummaty}} devrait être appliqué, selon nous, pour des
raisons de durée d'exécution à un résultat final.
\end{abstract}
\begin{itemize}
\tightlist
\item
  \href{https://jtools.jacob-long.com/index.html}{Documentation sur le
  package \textbf{jtools} (Jacob Long)}
\item
  A ce jour, la maintenance du package est assurée {[}version 2.2.1 à
  juin 2023{]}
\item
  Les nombre de modèles pris en charge par le package est
  malheureusement assez réduit. On notera cependant la prise en charge
  de la fonction \texttt{svyglm} du package survey.
\item
  Les fonctionnalités d'exportation ne seront pas traitées ici.
\end{itemize}

\hypertarget{installation}{%
\section{\texorpdfstring{\textbf{Installation}}{Installation}}\label{installation}}

\begin{Shaded}
\begin{Highlighting}[]
\FunctionTok{install.packages}\NormalTok{(}\StringTok{"jtools"}\NormalTok{)}
\end{Highlighting}
\end{Shaded}

ou

\begin{Shaded}
\begin{Highlighting}[]
\FunctionTok{install.packages}\NormalTok{(}\StringTok{"devtools"}\NormalTok{)}
\NormalTok{devtools}\SpecialCharTok{::}\FunctionTok{install\_github}\NormalTok{(}\StringTok{"jacob{-}long/jtools"}\NormalTok{)}
\end{Highlighting}
\end{Shaded}

\hypertarget{syntaxe-de-la-fonction-summ}{%
\section{\texorpdfstring{\textbf{Syntaxe de la fonction
\texttt{summ()}}}{Syntaxe de la fonction summ()}}\label{syntaxe-de-la-fonction-summ}}

La syntaxe est particulièrement simple, elle consiste juste à appliquée
à la fonction \texttt{summ()} l'objet généré par la régression. Quelques
options comme \texttt{digits}, \texttt{confint}, \texttt{exp} permettent
d'améliorer et enrichir l'output.

\begin{codelisting}

\caption{\texttt{syntaxe minimale}}

\begin{Shaded}
\begin{Highlighting}[]
\NormalTok{fit }\OtherTok{=} \FunctionTok{lm}\NormalTok{(y }\SpecialCharTok{\textasciitilde{}}\NormalTok{ x , }\AttributeTok{data=}\NormalTok{df)}
\FunctionTok{summ}\NormalTok{(fit)}
\end{Highlighting}
\end{Shaded}

\end{codelisting}

\hypertarget{exemples-1}{%
\section{\texorpdfstring{\textbf{Exemples
\footnote{mesure de la temps artérielle (\texttt{lm}) et du risque
  d'hypertension (\texttt{glm} et \texttt{svyglm})}}}{Exemples }}\label{exemples-1}}

\hypertarget{avec-la-fonction-lm}{%
\subsection{\texorpdfstring{Avec la fonction
\texttt{lm()}}{Avec la fonction lm()}}\label{avec-la-fonction-lm}}

\href{https://jtools.jacob-long.com/reference/summ.lm.html}{Liste des
options}

\hypertarget{annotated-cell-4}{%
\label{annotated-cell-4}}%
\begin{Shaded}
\begin{Highlighting}[]
\FunctionTok{library}\NormalTok{(jtools)}
\FunctionTok{library}\NormalTok{(readr)}

\NormalTok{df }\OtherTok{=}  \FunctionTok{read.csv}\NormalTok{(}\StringTok{"https://raw.githubusercontent.com/mthevenin/intro\_logit/main/hypertension2.csv"}\NormalTok{)}

\NormalTok{fit }\OtherTok{=} \FunctionTok{lm}\NormalTok{(bpsystol }\SpecialCharTok{\textasciitilde{}}\NormalTok{ age }\SpecialCharTok{+} \FunctionTok{I}\NormalTok{(sex) }\SpecialCharTok{+} \FunctionTok{I}\NormalTok{(black) }\SpecialCharTok{+} \FunctionTok{I}\NormalTok{(region), }\AttributeTok{data=}\NormalTok{df)}

\FunctionTok{summ}\NormalTok{(fit, }\AttributeTok{digits=}\DecValTok{4}\NormalTok{)  }\CommentTok{\#\textless{}1\textgreater{}                                                       }
\end{Highlighting}
\end{Shaded}

\begin{description}
\tightlist
\item[\circled{1}]
digits=4 =\textgreater{} Les résultats sont reportés avec 4 décimales
\end{description}

\begin{verbatim}
MODEL INFO:
Observations: 10351
Dependent Variable: bpsystol
Type: OLS linear regression 

MODEL FIT:
F(6,10344) = 552.2494, p = 0.0000
R² = 0.2426
Adj. R² = 0.2422 

Standard errors: OLS
---------------------------------------------------------------
                              Est.     S.E.     t val.        p
----------------------- ---------- -------- ---------- --------
(Intercept)               102.2001   0.8979   113.8168   0.0000
age                         0.6563   0.0116    56.5104   0.0000
I(sex)Male                  4.0350   0.3999    10.0911   0.0000
I(black)Not Black          -4.6494   0.6640    -7.0019   0.0000
I(region)NE                 0.2570   0.5892     0.4362   0.6627
I(region)S                 -0.7920   0.5450    -1.4532   0.1462
I(region)W                 -0.5221   0.5543    -0.9420   0.3462
---------------------------------------------------------------
\end{verbatim}

\hypertarget{avec-la-fonction-glm-lien-logit}{%
\subsection{\texorpdfstring{Avec la fonction \texttt{glm()} {[}lien
logit{]}}{Avec la fonction glm() {[}lien logit{]}}}\label{avec-la-fonction-glm-lien-logit}}

\href{https://jtools.jacob-long.com/reference/summ.glm.html}{Liste des
options}

\hypertarget{annotated-cell-5}{%
\label{annotated-cell-5}}%
\begin{Shaded}
\begin{Highlighting}[]
\NormalTok{fit }\OtherTok{=} \FunctionTok{glm}\NormalTok{(highbp }\SpecialCharTok{\textasciitilde{}}\NormalTok{ age }\SpecialCharTok{+} \FunctionTok{I}\NormalTok{(sex) }\SpecialCharTok{+} \FunctionTok{I}\NormalTok{(black) }\SpecialCharTok{+} \FunctionTok{I}\NormalTok{(region), }\AttributeTok{family=}\NormalTok{binomial, }\AttributeTok{data=}\NormalTok{df)   }

\FunctionTok{summ}\NormalTok{(fit, }\AttributeTok{digits=}\DecValTok{4}\NormalTok{, }\AttributeTok{confint=}\ConstantTok{TRUE}\NormalTok{, }\AttributeTok{exp=}\ConstantTok{TRUE}\NormalTok{)                                           }\CommentTok{\#\textless{}1\textgreater{}}
\end{Highlighting}
\end{Shaded}

\begin{description}
\tightlist
\item[\circled{1}]
On ajoute des intervalles de confiance (\texttt{confint=TRUE}) et le
report des estimateurs sous forme d'Odds Ratio (\texttt{exp=TRUE})
\end{description}

\begin{verbatim}
MODEL INFO:
Observations: 10351
Dependent Variable: highbp
Type: Generalized linear model
  Family: binomial 
  Link function: logit 

MODEL FIT:
χ²(6) = 1623.1601, p = 0.0000
Pseudo-R² (Cragg-Uhler) = 0.1951
Pseudo-R² (McFadden) = 0.1151
AIC = 12492.3709, BIC = 12543.0848 

Standard errors: MLE
-------------------------------------------------------------------------
                          exp(Est.)     2.5%    97.5%     z val.        p
----------------------- ----------- -------- -------- ---------- --------
(Intercept)                  0.0853   0.0701   0.1038   -24.5720   0.0000
age                          1.0496   1.0469   1.0524    36.1832   0.0000
I(sex)Male                   1.5483   1.4223   1.6856    10.0890   0.0000
I(black)Not Black            0.5860   0.5092   0.6742    -7.4663   0.0000
I(region)NE                  1.1655   1.0287   1.3203     2.4051   0.0162
I(region)S                   1.0024   0.8930   1.1253     0.0414   0.9669
I(region)W                   1.0966   0.9746   1.2338     1.5322   0.1255
-------------------------------------------------------------------------
\end{verbatim}

\hypertarget{avec-la-fonction-svyglm-lien-logit}{%
\subsection{\texorpdfstring{Avec la fonction \texttt{svyglm()} {[}lien
logit{]}}{Avec la fonction svyglm() {[}lien logit{]}}}\label{avec-la-fonction-svyglm-lien-logit}}

\href{https://jtools.jacob-long.com/reference/summ.svyglm.html}{Liste
des options}

\begin{Shaded}
\begin{Highlighting}[]
\FunctionTok{library}\NormalTok{(survey)}

\NormalTok{w }\OtherTok{=} \FunctionTok{svydesign}\NormalTok{(}\AttributeTok{id=}\SpecialCharTok{\textasciitilde{}}\DecValTok{1}\NormalTok{, }\AttributeTok{weights=}\SpecialCharTok{\textasciitilde{}}\NormalTok{w, }\AttributeTok{data=}\NormalTok{df)}

\NormalTok{fit }\OtherTok{=} \FunctionTok{svyglm}\NormalTok{(highbp }\SpecialCharTok{\textasciitilde{}}\NormalTok{ age }\SpecialCharTok{+} \FunctionTok{I}\NormalTok{(sex) }\SpecialCharTok{+} \FunctionTok{I}\NormalTok{(black) }\SpecialCharTok{+} \FunctionTok{I}\NormalTok{(region), }\AttributeTok{family=}\NormalTok{binomial, }\AttributeTok{design=}\NormalTok{w)}
\FunctionTok{summ}\NormalTok{(fit, }\AttributeTok{digits=}\DecValTok{4}\NormalTok{, }\AttributeTok{confint=}\ConstantTok{TRUE}\NormalTok{, }\AttributeTok{exp=}\ConstantTok{TRUE}\NormalTok{)    }
\end{Highlighting}
\end{Shaded}

\begin{verbatim}
MODEL INFO:
Observations: 10351
Dependent Variable: highbp
Type: Analysis of complex survey design 
 Family: binomial 
 Link function: logit 

MODEL FIT:
Pseudo-R² (Cragg-Uhler) = 0.1881
Pseudo-R² (McFadden) = 0.1125
AIC = 12064.1922 

-------------------------------------------------------------------------
                          exp(Est.)     2.5%    97.5%     t val.        p
----------------------- ----------- -------- -------- ---------- --------
(Intercept)                  0.0692   0.0553   0.0865   -23.3763   0.0000
age                          1.0530   1.0497   1.0562    33.0767   0.0000
I(sex)Male                   1.8298   1.6529   2.0256    11.6459   0.0000
I(black)Not Black            0.5849   0.4913   0.6964    -6.0259   0.0000
I(region)NE                  1.1822   1.0245   1.3642     2.2906   0.0220
I(region)S                   0.9961   0.8670   1.1445    -0.0549   0.9562
I(region)W                   1.1225   0.9737   1.2940     1.5928   0.1112
-------------------------------------------------------------------------

Estimated dispersion parameter = 0.984 
\end{verbatim}



\end{document}
