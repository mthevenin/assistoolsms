% Options for packages loaded elsewhere
\PassOptionsToPackage{unicode}{hyperref}
\PassOptionsToPackage{hyphens}{url}
\PassOptionsToPackage{dvipsnames,svgnames,x11names}{xcolor}
%
\documentclass[
  letterpaper,
  DIV=11,
  numbers=noendperiod]{scrartcl}

\usepackage{amsmath,amssymb}
\usepackage{iftex}
\ifPDFTeX
  \usepackage[T1]{fontenc}
  \usepackage[utf8]{inputenc}
  \usepackage{textcomp} % provide euro and other symbols
\else % if luatex or xetex
  \usepackage{unicode-math}
  \defaultfontfeatures{Scale=MatchLowercase}
  \defaultfontfeatures[\rmfamily]{Ligatures=TeX,Scale=1}
\fi
\usepackage{lmodern}
\ifPDFTeX\else  
    % xetex/luatex font selection
\fi
% Use upquote if available, for straight quotes in verbatim environments
\IfFileExists{upquote.sty}{\usepackage{upquote}}{}
\IfFileExists{microtype.sty}{% use microtype if available
  \usepackage[]{microtype}
  \UseMicrotypeSet[protrusion]{basicmath} % disable protrusion for tt fonts
}{}
\makeatletter
\@ifundefined{KOMAClassName}{% if non-KOMA class
  \IfFileExists{parskip.sty}{%
    \usepackage{parskip}
  }{% else
    \setlength{\parindent}{0pt}
    \setlength{\parskip}{6pt plus 2pt minus 1pt}}
}{% if KOMA class
  \KOMAoptions{parskip=half}}
\makeatother
\usepackage{xcolor}
\usepackage[top=30mm,left=10mm,ight=10mm]{geometry}
\setlength{\emergencystretch}{3em} % prevent overfull lines
\setcounter{secnumdepth}{-\maxdimen} % remove section numbering
% Make \paragraph and \subparagraph free-standing
\ifx\paragraph\undefined\else
  \let\oldparagraph\paragraph
  \renewcommand{\paragraph}[1]{\oldparagraph{#1}\mbox{}}
\fi
\ifx\subparagraph\undefined\else
  \let\oldsubparagraph\subparagraph
  \renewcommand{\subparagraph}[1]{\oldsubparagraph{#1}\mbox{}}
\fi

\usepackage{color}
\usepackage{fancyvrb}
\newcommand{\VerbBar}{|}
\newcommand{\VERB}{\Verb[commandchars=\\\{\}]}
\DefineVerbatimEnvironment{Highlighting}{Verbatim}{commandchars=\\\{\}}
% Add ',fontsize=\small' for more characters per line
\usepackage{framed}
\definecolor{shadecolor}{RGB}{40,42,54}
\newenvironment{Shaded}{\begin{snugshade}}{\end{snugshade}}
\newcommand{\AlertTok}[1]{\textcolor[rgb]{1.00,0.33,0.33}{\textbf{#1}}}
\newcommand{\AnnotationTok}[1]{\textcolor[rgb]{1.00,0.47,0.78}{#1}}
\newcommand{\AttributeTok}[1]{\textcolor[rgb]{1.00,0.47,0.78}{#1}}
\newcommand{\BaseNTok}[1]{\textcolor[rgb]{0.74,0.58,0.98}{#1}}
\newcommand{\BuiltInTok}[1]{\textcolor[rgb]{0.55,0.91,0.99}{#1}}
\newcommand{\CharTok}[1]{\textcolor[rgb]{0.95,0.98,0.55}{#1}}
\newcommand{\CommentTok}[1]{\textcolor[rgb]{0.38,0.45,0.64}{#1}}
\newcommand{\CommentVarTok}[1]{\textcolor[rgb]{0.55,0.91,0.99}{#1}}
\newcommand{\ConstantTok}[1]{\textcolor[rgb]{0.74,0.58,0.98}{\textbf{#1}}}
\newcommand{\ControlFlowTok}[1]{\textcolor[rgb]{1.00,0.47,0.78}{#1}}
\newcommand{\DataTypeTok}[1]{\textcolor[rgb]{0.55,0.91,0.99}{\textit{#1}}}
\newcommand{\DecValTok}[1]{\textcolor[rgb]{0.74,0.58,0.98}{#1}}
\newcommand{\DocumentationTok}[1]{\textcolor[rgb]{1.00,0.72,0.42}{#1}}
\newcommand{\ErrorTok}[1]{\textcolor[rgb]{1.00,0.33,0.33}{\underline{#1}}}
\newcommand{\ExtensionTok}[1]{\textcolor[rgb]{0.55,0.91,0.99}{#1}}
\newcommand{\FloatTok}[1]{\textcolor[rgb]{0.74,0.58,0.98}{#1}}
\newcommand{\FunctionTok}[1]{\textcolor[rgb]{0.31,0.98,0.48}{#1}}
\newcommand{\ImportTok}[1]{\textcolor[rgb]{1.00,0.47,0.78}{#1}}
\newcommand{\InformationTok}[1]{\textcolor[rgb]{0.95,0.98,0.55}{#1}}
\newcommand{\KeywordTok}[1]{\textcolor[rgb]{1.00,0.47,0.78}{#1}}
\newcommand{\NormalTok}[1]{\textcolor[rgb]{0.97,0.97,0.95}{#1}}
\newcommand{\OperatorTok}[1]{\textcolor[rgb]{0.97,0.97,0.95}{#1}}
\newcommand{\OtherTok}[1]{\textcolor[rgb]{0.31,0.98,0.48}{#1}}
\newcommand{\PreprocessorTok}[1]{\textcolor[rgb]{1.00,0.47,0.78}{#1}}
\newcommand{\RegionMarkerTok}[1]{\textcolor[rgb]{0.55,0.91,0.99}{#1}}
\newcommand{\SpecialCharTok}[1]{\textcolor[rgb]{1.00,0.47,0.78}{#1}}
\newcommand{\SpecialStringTok}[1]{\textcolor[rgb]{0.95,0.98,0.55}{#1}}
\newcommand{\StringTok}[1]{\textcolor[rgb]{0.95,0.98,0.55}{#1}}
\newcommand{\VariableTok}[1]{\textcolor[rgb]{0.55,0.91,0.99}{#1}}
\newcommand{\VerbatimStringTok}[1]{\textcolor[rgb]{0.95,0.98,0.55}{#1}}
\newcommand{\WarningTok}[1]{\textcolor[rgb]{1.00,0.33,0.33}{#1}}

\providecommand{\tightlist}{%
  \setlength{\itemsep}{0pt}\setlength{\parskip}{0pt}}\usepackage{longtable,booktabs,array}
\usepackage{calc} % for calculating minipage widths
% Correct order of tables after \paragraph or \subparagraph
\usepackage{etoolbox}
\makeatletter
\patchcmd\longtable{\par}{\if@noskipsec\mbox{}\fi\par}{}{}
\makeatother
% Allow footnotes in longtable head/foot
\IfFileExists{footnotehyper.sty}{\usepackage{footnotehyper}}{\usepackage{footnote}}
\makesavenoteenv{longtable}
\usepackage{graphicx}
\makeatletter
\def\maxwidth{\ifdim\Gin@nat@width>\linewidth\linewidth\else\Gin@nat@width\fi}
\def\maxheight{\ifdim\Gin@nat@height>\textheight\textheight\else\Gin@nat@height\fi}
\makeatother
% Scale images if necessary, so that they will not overflow the page
% margins by default, and it is still possible to overwrite the defaults
% using explicit options in \includegraphics[width, height, ...]{}
\setkeys{Gin}{width=\maxwidth,height=\maxheight,keepaspectratio}
% Set default figure placement to htbp
\makeatletter
\def\fps@figure{htbp}
\makeatother

\KOMAoption{captions}{tableheading}
\makeatletter
\makeatother
\makeatletter
\makeatother
\makeatletter
\@ifpackageloaded{caption}{}{\usepackage{caption}}
\AtBeginDocument{%
\ifdefined\contentsname
  \renewcommand*\contentsname{Table des matières}
\else
  \newcommand\contentsname{Table des matières}
\fi
\ifdefined\listfigurename
  \renewcommand*\listfigurename{Liste des Figures}
\else
  \newcommand\listfigurename{Liste des Figures}
\fi
\ifdefined\listtablename
  \renewcommand*\listtablename{Liste des Tables}
\else
  \newcommand\listtablename{Liste des Tables}
\fi
\ifdefined\figurename
  \renewcommand*\figurename{Figure}
\else
  \newcommand\figurename{Figure}
\fi
\ifdefined\tablename
  \renewcommand*\tablename{Table}
\else
  \newcommand\tablename{Table}
\fi
}
\@ifpackageloaded{float}{}{\usepackage{float}}
\floatstyle{ruled}
\@ifundefined{c@chapter}{\newfloat{codelisting}{h}{lop}}{\newfloat{codelisting}{h}{lop}[chapter]}
\floatname{codelisting}{Listing}
\newcommand*\listoflistings{\listof{codelisting}{Liste des Listings}}
\makeatother
\makeatletter
\@ifpackageloaded{caption}{}{\usepackage{caption}}
\@ifpackageloaded{subcaption}{}{\usepackage{subcaption}}
\makeatother
\makeatletter
\makeatother
\ifLuaTeX
\usepackage[bidi=basic]{babel}
\else
\usepackage[bidi=default]{babel}
\fi
\babelprovide[main,import]{french}
% get rid of language-specific shorthands (see #6817):
\let\LanguageShortHands\languageshorthands
\def\languageshorthands#1{}
\ifLuaTeX
  \usepackage{selnolig}  % disable illegal ligatures
\fi
\IfFileExists{bookmark.sty}{\usepackage{bookmark}}{\usepackage{hyperref}}
\IfFileExists{xurl.sty}{\usepackage{xurl}}{} % add URL line breaks if available
\urlstyle{same} % disable monospaced font for URLs
\hypersetup{
  pdftitle={Récupérer et exécuter le test OLS de Grambsch-Therneau},
  pdfauthor={Marc Thévenin},
  pdflang={fr},
  colorlinks=true,
  linkcolor={blue},
  filecolor={Maroon},
  citecolor={Blue},
  urlcolor={Blue},
  pdfcreator={LaTeX via pandoc}}

\title{Récupérer et exécuter le test OLS de Grambsch-Therneau}
\author{Marc Thévenin}
\date{2023-06-13}

\begin{document}
\maketitle
\begin{abstract}
Comment récupérer la variante du test de Grambsch-Therneau implémentée
au package \texttt{survival} avant son passage à la v3. Permet de
s'assurer une reproductibilité avec les autres applications statistiques
(Stata, Sas, Python) avec des durées discrètes.
\end{abstract}
\textbf{Champ d'application}

\begin{itemize}
\tightlist
\item
  Modèle de Cox (analyse des durées)\\
\item
  Diagnostic sur l'hypothèse de proportionalité des risques
\end{itemize}

\textbf{Problématique}

\begin{itemize}
\tightlist
\item
  Depuis le passage à la version 3 du package \textbf{\texttt{survival}}
  en 2020, le test OLS sur les résidus de Schoenfeld a été supprimé et
  remplacé par le test GLS. Le second est la version originelle du test
  proposé en 1994 par P.Grambsch et T.Therneau.
\item
  En présence d'évènements simultanés (durée discrète), les résultats
  affichés par les deux versions peuvent fortement variés.
\item
  Lorsqu'on utilise le modèle de Cox avec des durées discrètes,
  situation très courante dans les sciences sociales, je préconise
  l'utilisation de la version dite \emph{simplifiée} (OLS).

  \begin{itemize}
  \tightlist
  \item
    Le modèle de Cox est une méthode à durée continue, les conditions de
    validité du test GLS ne sont établies avec des évènements mesurés
    simultanément.
  \item
    Contrainte de reproductibilité: \textbf{Stata}, \textbf{Sas},
    \textbf{Python} (\textbf{\texttt{lifelines}},
    \textbf{\texttt{statsmodels}})
  \end{itemize}
\end{itemize}

\textbf{Récupération et exécution du test OLS}

\begin{itemize}
\tightlist
\item
  J'ai récupéré le script de la fonction dans les archives du CRAN. Elle
  a simplement été renommée \textbf{\texttt{cox.zphold()}}.
\item
  Pour charger la fonction, il suffit d'appliquer
  \textbf{\texttt{source()}} qui joue le même rôle que
  \textbf{\texttt{library()}}.

  \begin{itemize}
  \tightlist
  \item
    Directement sur le dépôt github:

    \begin{itemize}
    \tightlist
    \item
      \textbf{\texttt{source("https://raw.githubusercontent.com/mthevenin/analyse\_duree/main/cox.zphold/cox.zphold.R")}}
    \end{itemize}
  \item
    Si le script a été enregistré en local:

    \begin{itemize}
    \tightlist
    \item
      \textbf{\texttt{source("local\_path/cox.zphold.R")}}
    \end{itemize}
  \end{itemize}
\end{itemize}

\textbf{Exemple}

\begin{codelisting}

\caption{\texttt{Récupération des données}}

\begin{Shaded}
\begin{Highlighting}[]
\FunctionTok{library}\NormalTok{(readr)}
\end{Highlighting}
\end{Shaded}

\end{codelisting}

\begin{verbatim}
Warning: le package 'readr' a été compilé avec la version R 4.2.3
\end{verbatim}

\begin{Shaded}
\begin{Highlighting}[]
\NormalTok{trans }\OtherTok{\textless{}{-}} \FunctionTok{read.csv}\NormalTok{(}\StringTok{"https://raw.githubusercontent.com/mthevenin/analyse\_duree/master/bases/transplantation.csv"}\NormalTok{)}
\end{Highlighting}
\end{Shaded}

\begin{codelisting}

\caption{\texttt{Estimation d'un modèle de Cox}}

\begin{Shaded}
\begin{Highlighting}[]
\FunctionTok{library}\NormalTok{(survival)}
\end{Highlighting}
\end{Shaded}

\end{codelisting}

\begin{verbatim}
Warning: le package 'survival' a été compilé avec la version R 4.2.3
\end{verbatim}

\begin{Shaded}
\begin{Highlighting}[]
\NormalTok{coxfit }\OtherTok{=} \FunctionTok{coxph}\NormalTok{(}\AttributeTok{formula =} \FunctionTok{Surv}\NormalTok{(stime, died) }\SpecialCharTok{\textasciitilde{}}\NormalTok{ year }\SpecialCharTok{+}\NormalTok{ age }\SpecialCharTok{+}\NormalTok{ surgery, }\AttributeTok{data =}\NormalTok{ trans)}
\end{Highlighting}
\end{Shaded}

\begin{codelisting}

\caption{\texttt{Réupération et exécution du test OLS}}

\begin{Shaded}
\begin{Highlighting}[]
\FunctionTok{source}\NormalTok{(}\StringTok{"https://raw.githubusercontent.com/mthevenin/analyse\_duree/main/cox.zphold/cox.zphold.R"}\NormalTok{)}
\FunctionTok{cox.zphold}\NormalTok{(coxfit)}
\end{Highlighting}
\end{Shaded}

\end{codelisting}

\begin{verbatim}
          rho chisq      p
year    0.159  1.96 0.1620
age     0.109  1.15 0.2845
surgery 0.251  3.96 0.0465
GLOBAL     NA  7.99 0.0462
\end{verbatim}



\end{document}
